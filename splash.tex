\documentclass[12pt]{article}

% You need graphics package
\usepackage{color}
\usepackage{graphicx}
\usepackage{tabularx}

\usepackage{pst-all}

\usepackage[annot]{advi}
\usepackage{hyperref}

%% \usepackage[annot]{advi} 
%% must occur after packages using PStricks macros 
%% (it redefines some of them) 

\definecolor{red}{rgb}{1.0,0.0,0.0}
\definecolor{blue}{rgb}{0.0,0.0,1.0}

\def\email#1{\textcolor{gray}{$<${#1}$>$ }}

\def\key#1{\textcolor{red}{#1}}
\def\ikey#1#2#3{\key{#1} & #2 & -- & #3\\}
\def\arg{\textcolor{blue}{arg }}

\pagestyle{empty}

\begin{document}
 
\newpage

\section*{Active dvi}

\begin{tabularx}{\linewidth}{Xl}
The presentation tool of choice for discriminating hackers.
See \texttt{http://pauillac.inria.fr/advi/}
&
\raisebox{-30pt}{\includegraphics{test/caml.eps}}
\end{tabularx}

\subsection*{Usage}
\texttt{advi [OPTIONS] DVIFILE}\\
Try \texttt{advi -help} for more information.

\subsection*{Key Bindings}

Advi recognizes the following keystrokes when typed in its window.
Each may optionally be preceded by a number, called \arg below, whose
interpretation is keystroke dependant. If \arg is unset, its value is
1, unless specified otherwise.

Advi maintains an history of previously visited pages organized as a stack. 
Additionnally, the history contains marks pages which are stronger than
unmarked pages. 

\vspace{\stretch{1}}
\noindent
\begin{tabularx}{\linewidth}{clcX}
\ikey{q}{quit}{End of show.}
\ikey{n}{next}
{Move \arg physical pages forward, leaving the history unchanged.}
\ikey{p}{previous}
{Move \arg physical pages backward, leaving the history unchanged.}
\ikey{,}{begin}{Move to the first page.}
\ikey{.}{end}{Move to the last page.}
\ikey{c}{center}{Center the page in the window, and resets the default resolution.}
\ikey{$<$}{smaller}{Scale the resolution by $1/1.414$}
\ikey{$>$}{bigger}{Scale the resolution by $1.414$}
\end{tabularx}

\vspace{\stretch{1}}

\begin{center}
Press \key{n} for next page.
\end{center}

\newpage

\Large
\begin{center}
\textbf{Key bindings}
\end{center}
\normalsize

\vspace{\stretch{1}}

\noindent
\begin{tabularx}{\linewidth}{clcX}
\ikey{return}{forward}
{Push the current page on the history stack, and move forward n physical page}
\ikey{tab}{mark and next}
{Push the current page on the history as marked, and move forward n physical page.}
\ikey{space}{continue}
{Move to the next pause if any, or do as \key{return} otherwise.}
\ikey{backspace}{back}
{Move \arg pages backward according to the history. The history stack is poped, accordingly.}
\ikey{escape}{find mark}
{Move \arg marked pages backward according to the history.
 Do nothing if the history does no contain any marked page.}
\ikey{g}{go}
{If \arg is unset move to the last page.
 If \arg is the current page do nothing.
 Otherwise, push the current page on the history as marked, and move
 to the physical page \arg.}
\ikey{f}{load fonts}{Load all the fonts used in the documents.  By default, fonts are loaded only when needed.}
\ikey{r}{redraw}{Redraw the current page.}
\ikey{R}{reload}{Reload the file and redraw the current page.}
\ikey{F}{make fonts}
{Does the same as \key{f}, and precomputes the glyphs of all characters used in the document.
This takes more time than loading the fonts, but the pages are drawn faster.}
\ikey{C}{clear}{Erase the image cache (buggy).}
\end{tabularx}

\vspace{\stretch{1}}

\newpage

\subsection*{Acknowledgments}

Authors and contributors are :

\begin{itemize}
\item Alexandre Miquel \email{Alexandre.Miquel@inria.fr}
\item Jun Furuse \email{Jun.Furuse@inria.fr}
\item Didier R\'emy \email{Didier.Remy@inria.fr}
\item Xavier Leroy \email{Xavier.Leroy@inria.fr}
\end{itemize}

\vspace{\stretch{1}}

\begin{center}
This program is distributed under the GNU LGPL. \\
See the enclosed file COPYING.
\end{center}

\vspace{\stretch{3}}

\end{document}
