
\documentclass [landscape,semlayer,semcolor,semlcmss,a4]{seminar}

\slidewidth 235mm
\slideheight 174mm
\centerslidesfalse

\usepackage[USenglish,francais]{babel}
\usepackage[latin1]{inputenc}
\usepackage[T1]{fontenc}

\usepackage{argv}
\defargv {ps2pdf}[]{
%  \rundriver{ps2pdf}
  \usepackage {times}
  \PassOptionsToPackage {psbubble}{xprosper}
  \PassOptionsToPackage {ignore}{advi}
}
\runargv

\usepackage {xprosper}
\AtLastOverlay
\AtEndSlide
  {\ifnum \value{overlaysCount}=0\relax
    \else
    \message {\theoverlaysCount/\thepause}
       \ifnum \value{overlaysCount}=\value{pause}\relax
       \advithumbnail[\theslide.\theoverlaysCount]\relax 
       \fi
   \fi}


\usepackage {amssymb}
\usepackage {color}
\definecolor{lightred}{rgb}{1,0.8,0.8}
\definecolor{lightblue}{rgb}{0.7,0.7,1}
\definecolor{lightgreen}{rgb}{0.8,1,0.8}
\definecolor{purple}{rgb}{1,0,1}
\def\purple{\color{purple}}
\usepackage {pst-node}
\usepackage {hyperref}
\usepackage{shadow}


\newcommand{\centershabox}[1]{\begin{center}\shabox{#1}\end{center}}
\newcommand{\vcentershabox}[1]{\vfill\centershabox{#1}\vfill}




\def\advifooter{\vbox to 0em{\vbox to 8em {\vfill
\hfill{\adviembed[sticky=advianim,width=0.8cm,height=0.9cm]{animate -geometry !g! -window !p advilogo.anim.gif}}
} \vss}}


\let \Pause \pause
\let \pause \relax
\usepackage{advi}
\usepackage{advi-annot}
\let \pause \Pause

\slideframe {none}
\renewcommand{\slidetopmargin}{20pt}
\renewcommand{\slidebottommargin}{1cm}

\def\titlestyle{\color{lightblue}}
\newcommand{\slidetitle}[1]{\begingroup 
   \sl \bf \titlestyle {#1}
   \par
   \smallskip\hrule\medskip \endgroup} 

\newcommand{\redcol}[1]{\textcolor{red}{#1}}
\newcommand{\greencol}[1]{\textcolor{green}{#1}}
\newcommand{\bluecol}[1]{\textcolor{blue}{#1}}

\def \ebox #1{\colorbox{yellow}
   {\mbox {\large \strut #1}}}

\begin{document}

\pagestyle {empty}

\overlays*{\begin{slide}
\advibg[global]{image=world.jpg,fit=top}
\slidetitle{\Huge Logiciel Libre, \par Logiciel Propri\'etaire}
\vskip 2cm
\begin{center}
Roberto Di Cosmo\\
Universit\'e de Paris VII \& INRIA Roquencourt\\
\mbox{}\\
28 Novembre 2001
\end{center}
\advifooter
\end{slide}}

\advitransition{wipe}
\overlays*{\begin{slide}
\slidetitle{Quelques d\'efinitions}
\begin{description}
  \item[\bluecol{Freeware}]\mbox{}\\ logiciel gratuit\\\pause
  \item[\bluecol{Shareware}]\mbox{}\\ logiciel payant, mais avec une p\'eriode d'essai gratuite\\\pause
  \item[\bluecol{Free Software}]\mbox{}\\ (Open Source, Logiciel Libre)\\ Quelque chose
       de radicalement diff\'erent, \`a l'origine du succ\'es du Web et de l'Internet
\end{description}
\end{slide}}

\advitransition{block}
\overlays*{\begin{slide}
\slidetitle{Free, Open Source Software : Logiciel libre }
 \begin{description}
   \item[\greencol{Gratuit}] (anglais: free): \\\pause
        logiciel non payant (aujourd'hui)\pause
   \item[\bluecol{Libre}] (anglais: free): \\\pause
	logiciel avec 4 droits\pause
   \begin{itemize}
     \item Libert\'e d'\bluecol{utiliser} le logiciel\pause
     \item Libert\'e d'\bluecol{\'etudier} les sources du logiciel et de l'\bluecol{adapter} \`a ses besoins\pause
     \item Libert\'e de \bluecol{distribuer} des copies\pause
     \item Libert\'e de \bluecol{distribuer} les sources (\'eventuellement \bluecol{modifi\'ees})\\\pause
           \mbox{}\\
           Il y a des \redcol{obligations} aussi, qui  varient selon la licence: GPL/BSD/Mozilla/X, etc.
   \end{itemize}
 \end{description}
\end{slide}}

\advitransition{wipe}
\overlays*{\begin{slide}
\slidetitle{\bluecol{Libre} \emph{n'est pas} \greencol{gratuit} }
\vfill
  \begin{description}
    \item[\greencol{non libre}, \bluecol{gratuit}]: \mbox{}\\Internet Explorer, MacTCP, Acrobat Reader, freeware, etc.\pause
    \item[\greencol{non libre}, \bluecol{non gratuit}]: \mbox{}\\\pause le plus connu \ldots{}\pause\\
    \item[\greencol{libre},     \bluecol{gratuit}]: \mbox{}\\Mozilla, Linux, FreeBSD, OpenBSD, sendmail, perl, etc.\pause\\
    \item[\greencol{libre},     \bluecol{non gratuit}]: \mbox{}\\distributions commerciales de Linux, etc.
  \end{description}
\end{slide}}
\advitransition{block}
\overlays*{\begin{slide}
\slidetitle{Le logiciel libre \bluecol{respecte} le droit des auteurs}
  \begin{description}
    \item[n'est pas Napster]\mbox{}\\\pause
         L'auteur choisit \emph{librement} d'\'ecrire du logiciel libre\pause
    \item[n'est pas du ``domaine public'', ni ``libre de droits'']\mbox{}\\\pause
         L'auteur \emph{prot\`ege} la \emph{libert\'e} de son logiciel par une licence 
    \item[ne rel\`eve pas d'une ``logique d'abandon'']\mbox{}\\\pause
         L'auteur choisit une logique de valorisation innovante pour son logiciel
  \end{description}
\end{slide}}
\overlays*{\begin{slide}
\slidetitle{Le logiciel libre vs. le logiciel propri\'etaire}
  \begin{description}
    \item[Logiciel libre]:
    \begin{itemize}
      \item\bluecol{avantages p\'edagogiques ind\'eniables: acc\`es \`a une meilleure formation (\`a l'informatique)}\pause
      \item\bluecol{multiplie le nombre des programmeurs qui v\'erificateurs, divise les pirates}\pause:\pause l'acc\`es au code source attire les programmeurs comp\'etents
      \item\bluecol{redonne le contr\^ole aux utilisateurs}\pause
      \item\bluecol{permet d'\'echapper \`a la fuite en avant technologique}\pause
    \end{itemize}
    \item[Logiciel propri\'etaire]:\pause
    \begin{itemize}
      \item \redcol{ne permet pas d'adapter le logiciel, ni de le comprendre}\pause
      \item \redcol{aucun contr\^ole de l'\'evolution technologique}\pause
      \item \redcol{multiplie les pirates, divise les v\'erificateurs}\pause
      \item \redcol{forte tendence \`a la cr\'eation de monop\^oles qui l\`event une taxe sur l'information et rendent captifs les utilisateurs}
    \end{itemize}
  \end{description}
\end{slide}}
\advitransition{slide}
\overlays*{\begin{slide}
\slidetitle{\greencol{Le logiciel libre vs. le logiciel propri\'etaire}}
       \begin{description}
    \item[mod\`ele centr\'e sur les licences]: profit non proportionnel au travail, peu ou pas d'emplois\\\pause
          \`a la limite (ex: Microsoft) taxe monopoliste et mauvaise qualit\'e, \pause accumulation de richesse
          sans cr\'eation de richesse.
    \item[mod\`ele centr\'e sur les services]: tendence naturelle des grandes soci\'et\'es (IBM, Oracle etc.), \pause
         profit proportionnel au travail, beaucoup d'emplois qualifi\'es de proximit\'e.\\\pause
         \`a la limite, le Logiciel Libre\pause

  \end{description}
\pause
  \begin{center}
        \begin{tabular}{|c|r|r|r|r|c|c|}
         \hline & Income & \greencol{P}rofit & \% & \bluecol{E}mployees & \fromPause[0]{\adviannot{$\greencol{P}/\bluecol{E}$}(-2,1){``taxe'' collect\'e par employ\'e}}& \fromPause[4]{\adviannot{$\bluecol{E}/\greencol{P}$}(-3,1){employ\'es \`a \emph{votre} service}}\\\hline
         IBM & 81,667 M\$ & 6,328 M\$       & 7  & 290.000 &  \fromPause[1]{ 21820 \$} & \fromPause[4]{45}\\\hline
         Oracle & 7,143 M\$ & 955 M\$       & 13 &  40.000 &  \fromPause[2]{ 23875 \$} & \fromPause[5]{41}\\\hline
         Microsoft & 20,000 M\$ & 8,000 M\$ & 40 &  29.000 &  \fromPause[3]{275000 \$} & \fromPause[6]{ 3}\\\hline
        \end{tabular}
  \end{center}
\pause
\pause
\pause
\pause
\pause
\pause
\end{slide}}
\advitransition{wipe}
\overlays*{\begin{slide}
\slidetitle{Les d\'erives du logiciel propri\'etaire}
  \begin{description}
      \item[la s\'ecurit\'e impossible]: \mbox{}\\\pause Sans code source, vous
           ne pouvez pas modifier un logiciel.\\\pause Mais l'utilisateur \emph{veut}
           personnaliser les logiciels!\\\pause 
           Voila donc appara\^{\i}tre \adviannot{VisualBasic}(1,2){Melissa, ILoveYou etc.},
           \adviannot{ActiveX}(1,2){CCC: vol sur les comptes en ligne}, etc.
           qui ouvrent des \adviannot{trous b\'eants de s\'ecurit\'e}(1,1){RSA!}.\\
           \pause
           Combien co\^utent ces virus \`a l'\'Etat chaque ann\'ee? \pause
      \item[les formats opaques]: \mbox{}\\\pause Le workflow vaseux \ldots{}
      \item[le mod\`ele de licences variable]:\mbox{}\\\pause Windows 
            et Office \redcol{\adviannot{X}(-1,1){eXPerience?}\adviannot{P}(1,1){eXPires!}} 
            sont arriv\'es!\pause
      \item[\ldots{}]: \ldots{}
  \end{description}
\end{slide}}
\advitransition{none}
\overlays*{\begin{slide}
\slidetitle{Le logiciel propri\'etaire dans le contexte acad\'emique: I}
L'Universit\'e en tant que \greencol{utilisatrice} de logiciel.\\
  \begin{description}
    \item[Co\^ut \'el\'ev\'e]: \mbox{}\\\pause
         les logiciels propri\'etaires sont vendus \`a prix exorbitant aux
         Universit\'es (ex: les Windows/Office pre-install\'es)\pause
    \item[Mauvaise p\'edagogie]: \mbox{}\\\pause
         pas d'exercice d'esprit critique, limites arbitraires des
         \adviannot{connaissances}(2,1){``conna\^{\i}tre c'est savoir faire'', selon Vico (\emph{circa} 1700)} qu'un \'etudiant peut obtenir\pause
    \item[Violation du devoir d'impartialit\'e]: \mbox{}\\\pause
         on doit utiliser \bluecol{des} logiciels pour former,
         et non pas former \`a utiliser \redcol{un} logiciel\pause
    \item[Philosophie radicalement contraire \`a l'esprit acad\'emique]:\mbox{}\\\pause
         Programme \emph{Comp\'etence 2000}, notes de cours, polycopi\'es etc.
  \end{description}
\end{slide}}
\overlays*{\begin{slide}
\slidetitle{Le logiciel propri\'etaire dans le contexte acad\'emique: II}
L'Universit\'e en tant que \bluecol{cr\'eatrice} de logiciel.
  \begin{description}
    \item[Entrave \`a la recherce]: \mbox{}\\des 
         \adviannot{licences restrictives}(2,-1.5){\begin{minipage}{10em}
                                              On peut dire de m\^eme pour le copyright 
                                              sur les articles impos\'es par la plupart 
                                              des revues!
                                                \end{minipage}} sont incompatibles
         avec l'activit\'e de recherche et coop\'eration scientifique.
    \item[Nature des logiciels]: \mbox{}\\une tr\`es grande quantit\'e de logiciel
         d\'ev\'elopp\'e dans le cadre acad\'emique n'est pas industrialisable 
         \adviannot{facilement}(-2,2){\begin{minipage}{20em}
                                       Pour ceux qui le sont, une licence restrictive
                                       les prive des contributions pr\'ecieuses de la communaut\'e
                                       \end{minipage}}
    \item[Entrave \`a la participation des chercheurs au projets libres]:\mbox{}\\
         un enseignant, chercheur ou technicien est port\'e par sa nature \`a coop\'erer
         \`a des projets libres, dont l'Institution b\'en\'eficie en tant qu'utilisatrice. 
         Des licences restrictives seraient contreproductives et priveraient l'Institution
         de retours en image \adviannot{importants}(1,2){\begin{minipage}{15em}Remy Card, Xavier Leroy, Jer\^ome Vouillon, Yves Legrandgerard, Juliusz Chroboczeck, etc \ldots{}\end{minipage}}. 
  \end{description}
\end{slide}}
\overlays*{\begin{slide}
\slidetitle{Le logiciel libre dans le contexte acad\'emique}
L'Universit\'e en tant qu'\bluecol{utilisatrice} de logiciel.
  \begin{description}
    \item[\adviannot{\'egalit\'e des chances}(1,1){protection de la propri\'et\'e intellectuelle}]: \mbox{}\\\pause des projets comme
         Artouste ou DemoLinux permettent \`a \emph{tous} les \'etudiants d'acc\'eder aux outils informatiques
         enseign\'es \`a l'Universit\'e \pause sans \adviannot{passer dans l'illegalit\'e}(-2,-1.5){\begin{minipage}{15em}
                                              Combien de copies pirates de Mathematica, Maple,
                                              Word, Excel, Visual C++ etc.?
                                                \end{minipage}} 
    \item[adaptation aux exigences p\'edagogiques]: \mbox{}\\\pause on peut choisir et adapter \`a ses besoins
         pedagogiques les logiciels libres
         \adviannot{\`a ses besoins}(1,1){\begin{minipage}{10em}
                                       minilangage g--\\
                                       projets de compilation\\
                                       \ldots{}
                                       \end{minipage}}
    \item[ma\^{\i}trise du parc logiciel et mat\'eriel]:\mbox{}\\\pause
         p\'erennit\'e des choix logiciels,\pause plus longue dur\'ee du mat\'eriel,\pause
         meilleure image vis \`a vis des \'etudiants, \`a parit\'e de budget
    \item[meilleure s\'ecurit\'e]:\mbox{}\\\pause
         reduction de l'impact des macro-virus
  \end{description}
\end{slide}}
\overlays*{\begin{slide}
\slidetitle{Une solution adapt\'ee, \`a l'\'etude \`a Paris VII}
  \begin{description}
    \item[conna\^{\i}tre] les logiciels libres\\\pause cette journ\'ee\pause
    \item[reconna\^{\i}tre] les logiciels libres\\\pause une entr\'ee dans la charte de propri\'et\'e intellectuelle\pause
    \item[recenser] les logiciels libres\\\pause un projet \`a long terme, qui n\'ecessite des moyens\pause
    \item[valoriser] les logiciels libres\\\pause sensibilisation des instances comp\'etentes\pause
    \item[aider] les logiciels libres\\\pause infrastructure et support soutenus officiellement
  \end{description}
  \pause
  \adviannot{Un remerciement particulier au BVRI}(1,1){pour sa clairvoyance}
\end{slide}}

\overlays*{\begin{slide}
\slidetitle{Il reste du chemin \`a faire: quelques pistes}
  \begin{description}
    \item[financement des logiciels libres] libre n'est pas gratuit!\\\pause est-il avis\'e de gaspiller des millions de francs en
         \adviannot{licences}(-1,1){bien des fois inutiles!} et en personnel \adviannot{pr\^et\'e}(1,1){Conventions Select \ldots{}} aux 
         fournisseurs propri\'etaires d'un c\^ot\'e,\pause et refuser tout investissement dans l'\'equivalent libre?\pause
    \item[recyclage du mat\'eriel] les logiciels libres permettent d'utiliser le mat\'eriel plus longtemps\\\pause 
         ne doit-on pas \adviannot{faire un effort de recyclage}(-1,1){les entreprises sont en avance}, notamment envers \'ecoles 
         et associations?\pause
    \item[\ldots{}] \ldots{}
  \end{description}
\end{slide}}
\overlays*{\begin{slide}
\slidetitle{Des questions?}
  \begin{description}
    \item[est-ce du PowerPoint] ? \\\pause \adviannot{Non}(1,1){c'est {\ActiveDVI}!}\pause
    \item[qu'est-ce que {\ActiveDVI}] ? \pause un visualiseur DVI \'ecrit enti\`erement en Ocaml \`a l'INRIA\ldots{}
    \item[\ldots{}] \ldots{}
  \end{description}
\end{slide}}
\end{document}
