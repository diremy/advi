%; tx     tx $*.tex
%; dview  dview -mlslides $*.dvi &
%; whizzy  slide -dvi "dview -mlslides."

\documentclass [landscape,semlayer,semcolor,semlcmss]{seminar}

\def \prsl {\green}
\usepackage {xprosper}
\def \jumpto #1{\special{html:<A href="jumpto:#1">}\special{html:</A>}}
\def \here #1{\hypertarget{#1}{}}

% \RequirePackage{semhelv}
% \ptsize{14}

%\input semlcmss.sty
%\usefont{T1}{phv}{m}{n}\fontsize{14.4pt}{12pt}\selectfont
%\usefont{T1}{phv}{m}{n}\fontsize{14.4pt}{14pt}\selectfont


\usepackage {amssymb}
\usepackage {pst-node}
\usepackage {hyperref}

%%% colors
\usepackage {color}
\definecolor{lightblue}{rgb}{0.7,0.7,1}
\definecolor{lightred}{rgb}{1,0.8,0.8}
\definecolor{lightgreen}{rgb}{0.6,1,0.6}


%%% both prosper and advi define the same macro pause, to be fixed

\let \Pause \pause
\let \pause \relax
\usepackage {advi}
\let \pause \Pause

\def \subpar #1{\par\medskip {\blue\bf #1}}
\let \partitle \subpar
\def \programcolor {\red\rm}
\def \jc {\@\triangleright\@}
\def \jcand {\@\&\@}

%\twoup
%\slidesmag{1}


%\slideframe {plain}\slideframewidth 0.1mm\slideframesep  4mm 
\slideframe {none}
%\renewcommand{\slideleftmargin}{-1cm}
\renewcommand{\slidetopmargin}{20pt}
\renewcommand{\slidebottommargin}{1cm}

\iftrue
\newpagestyle {mystyle}{}{}%{\hfil {\tiny \thepage}}
\pagestyle {mystyle}
\fi

\def \tf {\blue \bf}


%\pagestyle {plain}

\makeatletter 
\def \verbatim@font {\tt\red}
\makeatother


\iffalse
 \let \progfont  \rm
 \let \sltt \sl

 \let \progfont \tt
  \font \progfont  cmss10 scaled 2000
  \font \sltt  cmssi10 scaled 2000

 \def \progsize {\normalsize \progfont\baselineskip 0.0pt}
 \def \progstyle {\progsize \progfont }

 \def \progstyle {\progsize \progfont  \progcolor}
 \let \progcolor \black
 \def \reset {\progfont \progcolor}
 \def \reply {\sltt \red}
\fi

\let  ~ \hfill

\def \slidetitle #1{\begingroup 
   \sl \bf \purple {#1}\hfill {\tiny \thepage}\hbox{}\par
   \smallskip\hrule\medskip \endgroup} 
\def \titlestyle {\color{cyan}}

\renewcommand{\slidetitle}[1]{\vskip 2ex \subsection* 
{\centerline {\titlestyle \Large #1}}
}
\let \sameslide \newslide

\let \l \ell
\let \r \rho
\def \ebox #1{\colorbox{yellow}
   {\mbox {\large \strut #1}}}
\def \ie.{{\em i.e.}}
\let \@\;
\def \mathprefix #1{\mathopen{}\mathrel{\tt {#1}}}

\begin{document}

\pagestyle {empty}

%%%%%%%%%%%%%%%%%%%%%%%%%%%%%%%%%%%%%%%%%%%%%%%%%%%%%%%%%%%%%%%%%%%%%%%%%%%%
%%%%%%%%%%%%%%%%%%%%%%%%%%%%%%%%%%%%%%%%%%%%%%%%%%%%%%%%%%%%%%%%%%%%%%%%%%%%
\pagestyle {mystyle}
\def \title #1{\paragraph* {\tf #1}}
%%%%%%%%%%%%%%%%%%%%%%%%%%%%%%%%%%%%%%%%%%%%%%%%%%%%%%%%%%%%%%%%%%%%%%%%%%%%
%%%%%%%%%%%%%%%%%%%%%%%%%%%%%%%%%%%%%%%%%%%%%%%%%%%%%%%%%%%%%%%%%%%%%%%%%%%%
\overlays*{\begin{slide}
\slidetitle{Effects with advi}

\large

A concurrent, object-oriented calculus with:
\vspace {0.5em}
\begin {itemize}
\item 
\ebox {\bf \red object internal concurrency}

\begin{Advirecord}{1}%
--- not just sequential objects running concurrently!
\end{Advirecord}

\item
\ebox {\bf \red incremental assembling of object definitions}

\adviwait 
\adviplay{1}

\begin{Advirecord}{2}
 --- including the refinement of both
\begin {itemize}
\item behavior ~{\sl traditional in object-orientation}
\item synchronization ~{\sl expected in a concurrent setting}
\end {itemize}
\end{Advirecord}

\adviwait\adviplay{2}
\end {itemize}

\end{slide}}
%%%%%%%%%%%%%%%%%%%%%%%%%%%%%%%%%%%%%%%%%%%%%%%%%%%%%%%%%%%%%%%%%%%%%%%%%%%%
\overlays*{\begin{slide}
\slidetitle{Same effects with prosper}

\large

A concurrent, object-oriented calculus with:
\vspace {0.5em}
\begin {itemize}
\item 
\ebox {\bf \red object internal concurrency}

\fromPause[1]{
--- not just sequential objects running concurrently!
}{}

\item
\ebox {\bf \red incremental assembling of object definitions}

\pause[2]
 --- including the refinement of both
\begin {itemize}
\item behavior ~{\sl traditional in object-orientation}
\item synchronization ~{\sl expected in a concurrent setting}
\end {itemize}
\end {itemize}

\end{slide}}
%%%%%%%%%%%%%%%%%%%%%%%%%%%%%%%%%%%%%%%%%%%%%%%%%%%%%%%%%%%%%%%%%%%%%%%%%%%%
\overlays*{\begin{slide}
\slidetitle {Definition of behaviors in the $\pi$-calculus}
%\jumpto {after-pi}

\large
$$
\def \u{\mbox {\large \bf u\hskip 0.3ex}}
\def \z{\mbox {\large \bf z\hskip 0.3ex}}
\begin{array}{@{}ccccc}
&& 
\bulle %[=1]
    {\ovalnode* [fillstyle=solid,fillcolor=lightred]{u1}
        {\u?(x)\,P}}(0.2,-1.5){\mbox{Reception on $u$}} & 
\\
\hfil
\bulle %[=1]
  {\ovalnode* [fillstyle=solid,fillcolor=lightred]{u2}
        {\mathprefix {rec}{\u?(x) \, Q}}}
  (-1,-2)
  {\begin{minipage}{7em}Replicated \\reception on $u$\end{minipage}}&  
\\
& 
\bulle %[=1]
  {\ovalnode* [fillstyle=solid,fillcolor=lightgreen]{u3}
        {\hbox{$\u!(v)$}}}
   (2,-2)
   {\mbox{Emission on $u$}}
%\pause
& 
&& 
\fromPause[1]{\ovalnode* [fillstyle=solid,fillcolor=lightgreen]{u3b}
        {\hbox{$\u!(w)$}}}
\\[2ex]
\ifpause[=3]{\CancelColors}{}
\betweenPauses23
  {\ovalnode* [fillstyle=solid,fillcolor=lightblue]{u5}
        {\hbox{$dyn!(\psframebox*[fillstyle=solid,fillcolor=lightred]{\u})$}}
}
\RestoreColors
\hfill
\\
&
\betweenPauses23{\rnode{u45}{\color{blue}\star}}& 
\fromPause[3]{\ovalnode* [fillstyle=solid,fillcolor=lightred]{u6}{\u?(x) \, S}}
\\
\ifpause[=3]{\CancelColors}{}
\betweenPauses23{\ovalnode* [fillstyle=solid,fillcolor=lightblue]{u4}
  {dyn?(\psframebox*[fillstyle=solid,fillcolor=yellow]{\z}) 
        \psovalbox*[fillstyle=solid,fillcolor=lightred]
          {{\psframebox*[fillcolor=yellow]{\z}}{?(x) \,S}}}}
\hskip -3em
\RestoreColors
\\
\end{array}
\ncarc[linestyle=dashed,linecolor=red]{-}{u2}{u1}
\ncarc[linecolor=green]{-}{u3}{u1}
\ncarc[linecolor=green]{-}{u2}{u3}
\fromPause*[1]{
\ncarc[linecolor=green]{-}{u3b}{u1}
\ncarc[linecolor=green]{-}{u2}{u3b}
}
\betweenPauses*23{
\ncarc[linecolor=green]{-}{u45}{u4}
\ncarc[linecolor=green]{-}{u5}{u45}
}
\fromPause*[4]{
\ncarc[linecolor=green]{-}{u3}{u6}
\ncarc[linecolor=green]{-}{u3b}{u6}
\ncarc[linestyle=dashed,linecolor=red]{-}{u6}{u2}
\ncarc[linestyle=dashed,linecolor=red]{-}{u1}{u6}
}
\onlyPause*[3]{
\ncarc[linecolor=lightblue,doubleline=true]{<-}{u6}{u45}
}
$$
\vskip -6em
\null
\pause[4]

\large
\slidetitle{\leftline {\fromPause[2]{\ebox{Too loose!}}}}
\begin {itemize}
\pause
\item
{\ifpause[>0]{\color{gray}}{}
Behaviors attached to a name are not syntactically
known; they are reconfigurable, dynamically.}

\item
{\ifpause[>0]{\color{gray}}{}
This makes reasoning {\em and} distributed computing difficult.}
\pause
\end {itemize}

\end{slide}}
%%%%%%%%%%%%%%%%%%%%%%%%%%%%%%%%%%%%%%%%%%%%%%%%%%%%%%%%%%%%%%%%%%%%%%%%%%%%
\here {after-pi}
\overlays*{\begin{slide}
%\jumpto {after-join}
\slidetitle {Definitions of behavior in the join-calculus}

\large
$$
\def \u{{\mbox {\large \bf u\hskip 0.3ex}}}
\def \w{{\mbox {\large \bf w\hskip 0.3ex}}}
\def \jc {{}\@\triangleright\@{}}
\def \jcand {\@\&\@}
\hfill
\begin{array}{ccccc}
\bulle%[=1]
{\rnode {u0}{\psframebox* [fillstyle=solid,fillcolor=lightred]
{\begin{array}{r@{}l} \u(x) \jc& P\\ \u(x) \jcand \w(y) \jc &Q\\
\end{array}}}}
(-0.5,-3)
{\mbox{\begin{minipage}{9em}
 Defines synchronizing names $\u$ and $\w$ 
 and their behaviors altogether
\end{minipage}}}
&
\hspace {4em}
\begin{array}{cc}
\bulle%[=1]
{\ovalnode* [fillstyle=solid,fillcolor=lightgreen]{u1}{\u!(v)}}
(1,-2)
{\mbox{\begin{minipage}{6em}Asynchronous emission on $\u$\end{minipage}}}
\pause
\\
& 
\fromPause[1]{\ovalnode* [fillstyle=solid,fillcolor=lightgreen]{u2}
{\w!(v)}}
\\
\end{array}
\\
\betweenPauses23{\circlenode*{?}{\ldots}} \hfil
&\fromPause[3]
{\ovalnode* [fillstyle=solid,fillcolor=lightgreen]{u3}{\u!(k)}}\hfil
\\
\betweenPauses23{
\hfill \circlenode* {s}{\color{blue}\star}
}
\\[1em]
\ifpause[=3]{\CancelColors}{}
\betweenPauses23{
\rnode {xy}{\psframebox* [fillstyle=solid,fillcolor=lightblue]
      {dyn(\psframebox* [fillcolor=yellow]{z}) \jc
      {\psovalbox* [fillstyle=solid,fillcolor=lightgreen]
      {\psframebox* [fillcolor=yellow]{z}!(k)}}
}}
}
\RestoreColors
\hfill 
&
\\
\end{array}
\ncarc[linecolor=green]{u0}{u1}
\fromPause*[1]{\ncarc[linecolor=green]{u2}{u0}}
\betweenPauses*23{
\ncarc[linecolor=lightblue]{s}{xy}
\ncarc[linecolor=lightblue]{?}{s}
}
\onlyPause*[3]{
\ncarc[linecolor=lightblue,doubleline=true]{<-}{u3}{s}
}
\fromPause*[4]{
\ncarc[linecolor=green]{u3}{u0}
}
$$
\vskip -3em
\pause[4]
\null
\begin {itemize}
\pause \item
{\ifpause[>0]{\color{gray}}{}
Name scoping is more constrained than in $\pi$-calculus}

\onlyPause*{\em
\begin {itemize}
\item
Behavior of a name is statically defined and lexically scoped. 

\item
Synchronizing names are jointly defined.
\end {itemize}}

\pause \item
{\ifpause[>0]{\color{gray}}{}
Easier for reasoning and  distributed computation}
\end {itemize}

\pause
\vspace {1em}
\centerline {\ebox {\LARGE {But it's too rigid!}}}

\end{slide}}
\here {after-join}

%%%%%%%%%%%%%%%%%%%%%%%%%%%%%%%%%%%%%%%%%%%%%%%%%%%%%%%%%%%%%%%%%%%%%%%%%%%%
%%%%%%%%%%%%%%%%%%%%%%%%%%%%%%%%%%%%%%%%%%%%%%%%%%%%%%%%%%%%%%%%%%%%%%%%%%%%

\end{document}
