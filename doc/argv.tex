\documentclass [12pt]{article}

\usepackage {manual}
\usepackage {fullpage}
\usepackage {makeidx}

\usepackage{argv}
\defargv {ps2pdf}[]{
  \rundriver{ps2pdf}
  \usepackage {times}
  \PassOptionsToPackage {ignore}{advi}
  \PassOptionsToPackage {psbubble}{xprosper}
}
\runargv

\title {{\ActiveDVI} manual}
\author {Didier R{\'{e}}my and ???}
\def \ActiveDVI {\textbf {Active-DVI}}


\begin{document}

\begin{abstract}
This package provides help for parameterizing the processing of latex source
file by comannd-line options.
\end{abstract}


\section {Calling latex}

The goal is to be able to call latex with different options, such as:
\begin{quote}
\verb+latex '\def\argv{dvips}\input' foo.tex+
\end{quote}
or
\begin{quote}
\verb+latex '\def\argv{ps2pdf}\input' foo.tex+
\end{quote}
and having different latex producing different outputs, so that they 
can be processed with {\tt dvips} or {\tt dvipdfm} shell commands.

Then, the file {\tt foo.tex} should look like:
\begin{quote}
\begin{verbatim}
\documentclass{article}

\usepackage{argv}
\def \driver {}
\defargv{dvips}[]{\def \driver{dvips}}
\defargv{dvips}[]{\def \driver{ps2pdf}}
\runargv

\usepackage [\driver]{hyperref}

\begin{document}
foo
\end{document}
\end{verbatim}
\end{quote}
Of course, in this simple example, some alternative, maybe shorter way could
have been found. However, one can do much more. Moreover, some options 
are actually predifined, and in this case, it would suffice to write:
\begin{quote}
\begin{verbatim}
\documentclass{article}
\usepackage[run]{argv}
\usepackage{hyperref}
\begin{document}
foo
\end{document}
\end{verbatim}
\end{quote}
and latex the file with either
\begin{quote}
\verb+latex '\def\argv{driver=dvips}\input' foo.tex+
\end{quote}
or
\begin{quote}
\verb+latex '\def\argv{driver=ps2pdf}\input' foo.tex+
\end{quote}

\section {Requirements}

The package uses the \doctt {keyval} package.

\section {New Commands}

\medskip\noindent
\docdef \defargv\docarg{name}\docopt{default}\docarg{value}
\begin{quote}
Define the option \docid{name} to execute \docid{value}, where value is the
body of a command with one argument (it may contain \verb"#1"). 
If \docid{default} is present, then the field \docid{name} can be called
without a value, and \docid{default} is used as the default value. 

In fact, the following names are predefined:

\medskip\noindent
\dockey {use}{package}
\begin{quote}
This will load the package \docid{package}
\end{quote}

\medskip\noindent
\dockey {driver}{driver}
\begin{quote}
This will execute \docdef\usrdriver\docarg{driver} as described below.
\end{quote}
\end{quote}

\medskip\noindent
\docdef \rundriver\docarg{driver}
\begin{quote}
If the \docid{driver} is recognize as a valid driver, it passed the
corresponding option to the two packages \doctt{hyperref} and
\doctt{graphicx}. The current drivers are:
\doctt{dvips}, \doctt{dvipdfm}, \doctt{ps2pdf}, and \doctt{pdflatex}.

(But it does not load the corresponding package).
\end{quote}

\medskip\noindent
\docdef \rundriver\docarg{driver}
\begin{quote}
If the \docid{driver} is recognize as a valid driver, it passed the
corresponding option to the two packages \doctt{hyperref} and
\doctt{graphicx}. The current drivers are:
\doctt{dvips}, \doctt{dvipdfm}, \doctt{ps2pdf}, and \doctt{pdflatex}.

(But it does not load the corresponding package).
\end{quote}

\medskip\noindent
\docdef \runargv
\begin{quote}
If \docdef \argv{} is defined (typically, on the command line), 
its contents is processed.
\end{quote}

Moreover, the package recognizes the option \doctt{run} which tells
the package to process arguments immediately. Hence,
\begin{quote}
\verb"\usepackage[run]{argv}"
\end{quote}
is equivalent to:
\begin{quote}
\verb"\usepackage{argv}\runargv"
\end{quote}


\end{document}
