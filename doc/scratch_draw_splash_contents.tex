\subsection*{Entering scratch drawing mode}

Press \key{S} to enter scratch drawing; the cursor is modified and
you must click somewhere on the page to start drawing
there. Before clicking, you can
\begin{itemize}
 \item press \key{?} to get help,
 \item press \key{\char94 G} to quit scratching immediately,
 \item press \key{Esc}
 to enter the scratch drawing settings mode and tune the color and
size of the pen.
\end{itemize}

\subsection*{Survival command kit when scratch drawing}

{\ActiveDVI} recognizes the following keystrokes when scratch drawing
on the page.

\noindent
\begin{tabularx}{\linewidth}{clcX}
\ikey{\char94 G}{quit}{End of scratch drawing.}
\ikey{Esc}{settings}{Enter the scratch drawing settings mode.}
\end{tabularx}

In the scratch drawing settings mode, the cursor is modified and you
can set some charateristics of the scratch drawing facility.


\subsection*{Scratch drawing settings mode keys}

When in the scratch drawing settings mode, the following keys have the
following respective meanings:

\subsubsection*{General scratch drawing settings keys}

\begin{itemize}
 \item press \key{?} to get help,
 \item press \key{Esc} to quit the settings mode,
 \item press \key{\char94 G} to quit scratching immediately.
\end{itemize}

\vspace{\stretch{1}}

\newpage

\subsubsection*{Setting the drawing line color}

\noindent
\begin{tabularx}{\linewidth}{clcX}
\ikey{b}{blue}{Set the color of the font to blue.}
\ikey{c}{cyan}{Set the color of the font to cyan.}
\ikey{g}{green}{Set the color of the font to green.}
\ikey{k}{black}{Set the color of the font to black.}
\ikey{m}{magenta}{Set the color of the font to magenta.}
\ikey{r}{red}{Set the color of the font to red.}
\ikey{w}{white}{Set the color of the font to white.}
\ikey{y}{yellow}{Set the color of the font to yellow.}
\ikey{B}{more blue}{Increment the blue component of the current color.}
\ikey{G}{more green}{Increment the green component of the current color.}
\ikey{R}{more red}{Increment the red component of the current color.}
\ikey{$+$}{positive increment}{Set the color increment to positive.}
\ikey{$-$}{negative increment}{Set the color increment to negative.}
\end{tabularx}

\subsubsection*{Setting the drawing line size}

\noindent
\begin{tabularx}{\linewidth}{clcX}
\ikey{$>$}{increment}{Increment by one the size of the line.}
\ikey{$<$}{decrement}{Decrement by one the size of the line.}
\end{tabularx}

\vspace{\stretch{1}}

\newpage

\subsubsection*{Setting the kind of figure to draw}

In the setting mode, pressing one of the following keys enter the (still
experimental) figure drawing mode:

\noindent
\begin{tabularx}{\linewidth}{clcX}
\ikey{V}{vertical line}{Draw a vertical line.}
\ikey{H}{horizontal line}{Draw a horizontal line.}
\ikey{S}{segment}{Draw a segment.}
\ikey{C}{circle}{Draw a circle.}
\ikey{p}{point}{Draw a point.}
\ikey{P}{polygone}{Draw a polygone.}
\ikey{e}{endpoly}{Close the polygone that is beeing drawn.}
\ikey{F}{free hand}{Draw a line following the pointer.}
\ikey{' '}{cancel}{Cancel the figure setting.}
\end{tabularx}

\vspace{\stretch{1}}
