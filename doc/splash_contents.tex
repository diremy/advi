Advi recognizes the keystrokes listed below when typed in its window.
Some keystrokes may optionally be preceded by a number, called \arg
below, whose interpretation is keystroke dependant. If \arg is unset,
its value is 1, unless specified otherwise.

Advi maintains an history of previously visited pages organized as a stack. 
Additionnally, the history contains marked pages which are stronger than
unmarked pages. 

\vspace{\stretch{1}}

\newpage

\vspace{\stretch{1}}

\subsection*{Survival command kit}

\noindent
\begin{tabularx}{\linewidth}{clcX}
\ikey{?}{info}{This quick info and key bindings help.}
\ikey{q}{quit}{End of show.}
\ikey{space}{continue}
{Move forward (\arg pauses forward if any, or do as \key{return} otherwise).}
\end{tabularx}

\vspace{\stretch{1}}

\subsection*{Moving to pages}

\noindent
\begin{tabularx}{\linewidth}{clcX}
\ikey{n}{next}
{Move \arg physical pages forward, leaving the history unchanged.}
\ikey{p}{previous}
{Move \arg physical pages backward, leaving the history unchanged.}
\ikey{,}{begin}{Move to the first page.}
\ikey{.}{end}{Move to the last page.}
\ikey{g}{go}
{If \arg is unset move to the last page.
 If \arg is the current page do nothing.
 Otherwise, push the current page on the history as marked, and move
 to physical page \arg.}
\end{tabularx}

\vspace{\stretch{1}}
\subsection*{Switching views}

\noindent
\begin{tabularx}{\linewidth}{clcX}
\ikey{\char94W}{switch}{Switch view between master and client (if any)}
\ikey{W}{autoswitch}{Toggle autoswitch flag}
\end{tabularx}

\vspace{\stretch{1}}

\subsection*{Table of contents}

\noindent
\begin{tabularx}{\linewidth}{clcX}
\ikey{T}{Thumbnails}{Process thumbnails.}
\ikey{t}{toc}{Display thumbnails if processed, or floating table of contents
 if available, or do nothing.}
\end{tabularx}

\vspace{\stretch{1}}

\vspace{\stretch{1}}

\newpage

\subsection*{Moving to pauses}

\noindent
\begin{tabularx}{\linewidth}{clcX}
\ikey{N}{next pause}{Move \arg pauses forward (equivalent to continue).}
\ikey{P}{previous pause}{Move \arg pauses backward.}
\end{tabularx}

\vspace{\stretch{1}}

\subsection*{Adjusting the page size}

\noindent
\begin{tabularx}{\linewidth}{clcX}
\ikey{\char94F}{fullscreen}{Adjust the size of the page to fit the
entire screen or reset the page to the default size (this is a toggle).}
\ikey{$<$}{smaller}{Scale down the resolution by scalestep (default
\tiny{$\sqrt{\sqrt{\sqrt 2}}$}).}
\ikey{$>$}{bigger}{Scale up the resolution by scalestep (default
\tiny{$\sqrt{\sqrt{\sqrt 2}}$}).}
\ikey{\char35}{fullpage}{Remove margins around the page and change
the resolution accordingly.}
\ikey{c}{center}{Center the page in the window, and resets the default
resolution.}
\end{tabularx}

\vspace{\stretch{1}}

\subsection*{Redisplay commands}

\noindent
\begin{tabularx}{\linewidth}{clcX}
\ikey{r}{redraw}{Redraw the current page to the current pause.}
\ikey{R}{reload}{Reload the file and redraw the current page.}
\ikey{\char94L}{redisplay}{Redisplay the current page to the first
pause of the page.}
\ikey{a}{active/passive}{toggle advi effects (so that reloading is silent).}
\end{tabularx}

\vspace{\stretch{1}}

\newpage

\vspace{\stretch{1}}

\subsection*{Moving the page in the window}

\noindent
\begin{tabularx}{\linewidth}{clcX}
\ikey{h}{page left}
{Moves one screen width toward the left of the page. Does nothing if the
  left part of the page is already displayed}
\ikey{l}{page right}
{Moves one screen width toward the right of the page. Does nothing if the
  right part of the page is already displayed}
\ikey{j}{page down}
{Moves one screen height toward the bottom of the page. Jumps to the top of
  next page, if there is one, and if the bottom of the page is already
  displayed.}
\ikey{k}{page up}
{Moves one screen height toward the top of the page. Jumps to the bottom
 previous page, if there is one, and if the top of the page is already
 displayed.}
\ikey{\char94 left button}{move page}
{A black line draws the page borders; moving the mouse then moves the
page in the window.}
\end{tabularx}

\vspace{\stretch{1}}

\newpage

\vspace{\stretch{1}}

\subsection*{Using the navigation history stack}

\noindent
\begin{tabularx}{\linewidth}{clcX}
\ikey{return}{forward}
{Push the current page on the history stack, and move forward n physical pages.}
\ikey{tab}{mark and next}
{Push the current page on the history as marked, and move forward n
physical pages.}
\ikey{backspace}{back}
{Move \arg pages backward according to the history. The history stack
is poped, accordingly.}
\ikey{escape}{find mark}
{Move \arg marked pages backward according to the history.
 Do nothing if the history does no contain any marked page.}
\end{tabularx}

\vspace{\stretch{1}}

\subsection*{Writing and drawing on the slide}

\noindent
\begin{tabularx}{\linewidth}{clcX}
\ikey{s}{scratch}{Give a pencil to type characters on the page.}
\ikey{S}{scratch}{Give a spray can to draw on the page.}
\end{tabularx}

\subsection*{Dealing with caches}

\noindent
\begin{tabularx}{\linewidth}{clcX}
\ikey{f}{load fonts}{Load all the fonts used in the document. By
default, fonts are loaded only when needed.}
\ikey{F}{make fonts}
{Does the same as \key{f}, and precomputes the glyphs of all the
characters used in the document.
This takes more time than loading the fonts, but the pages are drawn faster.}
\ikey{C}{clear}{Erase the image cache.}
\end{tabularx}

\vspace{\stretch{1}}





