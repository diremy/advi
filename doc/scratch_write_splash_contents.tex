Advi recognizes the keystrokes listed below when writing on the slide.

\subsection*{Survival command kit when scratching}

\noindent
\begin{tabularx}{\linewidth}{clcX}
\ikey{Escape}{quit}{End of scratching}
\ikey{\char94Q}{Scratch settings}{Enter to the setting mode where you can fix various scratching parameters (you can then enter '?' to get help).}
\ikey{\char94Z}{Scratch settings}{Similar to \char94Q.}
\end{tabularx}

In the normal scratching mode, pressing \char94Q or \char94Z
enter the setting mode. While in setting mode, the cursor is modified and
you can set some charateristics of the scratching mode. When in doubt,
enter '?' to get help about the proposed settings.

\vspace{\stretch{1}}

\subsection*{Setting the font color}

The following keys have the following respective meanings to change
the color of the font:

\noindent
\begin{tabularx}{\linewidth}{clcX}
\ikey{$?$}{help}{Give the list of settings available.}
\ikey{b}{blue}{Set the color of the font to blue.}
\ikey{c}{cyan}{Set the color of the font to cyan.}
\ikey{g}{green}{Set the color of the font to green.}
\ikey{k}{black}{Set the color of the font to black.}
\ikey{m}{magenta}{Set the color of the font to magenta.}
\ikey{r}{red}{Set the color of the font to red.}
\ikey{w}{white}{Set the color of the font to white.}
\ikey{y}{yellow}{Set the color of the font to yellow.}
\ikey{B}{more blue}{Increment the blue component of the color.}
\ikey{G}{more green}{Increment the green component of the current color.}
\ikey{R}{more red}{Increment the red component of the current color.}
\ikey{$+$}{positive increment}{Set the color increment to positive.}
\ikey{$-$}{negative increment}{Set the color increment to negative.}
\end{tabularx}

\subsection*{Setting the font size}

\noindent
\begin{tabularx}{\linewidth}{clcX}
\ikey{$>$}{increment}
{Increments the size of the font used to write on the slide.}
\ikey{$<$}{decrement}
{Decrements the size of the font used to write on the slide.}
\end{tabularx}

\vspace{\stretch{1}}





