\subsection*{Entering write scratching mode}

Press \key{s} to enter write scratching; the cursor is modified and
you then have to click somewhere on the page, to start writing text
there.

Before clicking to start writing,
\begin{citemize}
 \item press \key{?} to get help,
 \item press \key{\char94G} to quit scratching immediately,
 \item press \key{Esc}
 to enter the scratch write settings mode and tune the font and font size.
\end{citemize}

\subsection*{Survival command kit when scratching}

{\ActiveDVI} recognizes the following keystrokes when scratch writing
on the page.

\noindent
\begin{tabularx}{\linewidth}{clcX}
\ikey{\char94G}{quit}{End of scratching.}
\ikey{Esc}{settings}{Enter the write settings mode:
you can fix various scratching parameters, such as the font color or the font
size (press \key{?} to get help).}
\end{tabularx}

In the write scratching mode, press \key{Esc} to enter the
write settings mode. In the write settings mode, also press \key{Esc} to
quit the write settings mode. While in the write settings mode, the
cursor is modified and you can set some charateristics of the scratch
writing facility. When in doubt, press \key{?} to get help about the
proposed settings.

\vspace{\stretch{1}}

\newpage

\subsection*{Write settings mode keys}

When in the write settings mode, the following keys have the
following respective meanings:

\noindent
\begin{tabularx}{\linewidth}{clcX}
\ikey{$>$}{greater}{Increments the scratch font size.}
\ikey{$<$}{smaller}{Decrements the scratch font size.}
\ikey{b}{blue}{Set the color of the font to blue.}
\ikey{c}{cyan}{Set the color of the font to cyan.}
\ikey{g}{green}{Set the color of the font to green.}
\ikey{k}{black}{Set the color of the font to black.}
\ikey{m}{magenta}{Set the color of the font to magenta.}
\ikey{r}{red}{Set the color of the font to red.}
\ikey{w}{white}{Set the color of the font to white.}
\ikey{y}{yellow}{Set the color of the font to yellow.}
\ikey{B}{more blue}{Increment the blue component of the color.}
\ikey{G}{more green}{Increment the green component of the current color.}
\ikey{R}{more red}{Increment the red component of the current color.}
\ikey{$+$}{positive increment}{Set the color increment to positive.}
\ikey{$-$}{negative increment}{Set the color increment to negative.}
\ikey{$?$}{help}{Give the list of settings available.}
\ikey{Esc}{quit}{Quit the write settings mode.}
\end{tabularx}

\subsection*{Setting the scratching font size}

Just press \key{Esc} to enter the write settings mode, then
\key{$>$} or \key{$<$} to increment or decrement the font size; then
press \key{Esc} again, to leave the write settings mode and
continue to write on the page with the new font size.

\vspace{\stretch{1}}
