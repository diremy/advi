\documentclass{article}

\usepackage {manual}
\usepackage {makeidx}

\expandafter \ifx \csname driver\endcsname\relax \def \driver{}\fi 
\usepackage [\driver]{hyperref}
\usepackage {advi}
\usepackage {advi-annot}


\title {Active DVI manual}
\author {???}



\begin{document}

\maketitle

\begin{abstract}
Summary of Active DVI.
\end{abstract}

\tableofcontents

\section {Installation}


\section {Documentation}

\subsection {Pause, Record and Play}

\medskip\noindent
\docdef \adviwait\docopt{seconds}
\begin{quote}
Wait for \docid{seconds} or without argument forever, that is, until the
user requests to continue (or to change page).
\end{quote}

\medskip\noindent
\docdef \advirecord \doctt{[play]}\docarg{tag}\docarg{latex code}
\begin{quote}
The computes \docid{latex code} and record its DVI ouput under tag
\docid{tag}. The DVI output is not displayed during recording, unless the
option \doctt{play} is set.

Embedded recordings are prohibited. 
\end{quote}

\medskip\noindent
\docdef \adviplay\docopt{color}\docarg{tag}
\begin{quote}
Replay the DVI previously recorded as \docid{tag}.
The optional argument changes the color to \docid{color} during replay. 
\end{quote}

\subsection {Images}

\subsection {Backgrounds}

\medskip\noindent
\docdef \advisetbg\docarg{key=value list}
\begin{quote}
where \docid{key=value list} is a list of bindings of the followning kind:

\dockey{color}{color}
\begin{quote}
Set the background color to \docid{color}.
\end{quote}

\dockey{image}{file}
\begin{quote}
Use the file from \docid{file} as background.
\end{quote}

\dockey{alpha}{float}
\begin{quote}
Set the alpha factor for the background image to \docid{float}.
\end{quote}

\dockey{blend}{blend mode}
\begin{quote}
Set the blend mode to \docid{blend mode}, which should be one of the
following (unchecked): {\tt difference}, XXX?
\end{quote}



\end{quote}

\subsection {Transitions}

\docdef \advitransition \docopt{\doctt {global}}\docarg {key=val list}
\begin{quote}
where \docid{key=value list} is a list of bindings of the following kind:

\dockey{none}{} or \dockey{slide}{} or \dockey{block}{} or \dockey{wipe}{}
\begin{quote}
Set the transition mode to the corresponding key.
One of this key is mandatory (several is should be illegal, but in fact
overides the previous one). 
\end{quote}

\dockey{from}{direction}
\begin{quote}
Make the transition come from \docid{direction}.  Directions are not checked
but should be one of the following: 
\begin{center}
\tt
\begin{tabular}{l@{\qquad}c@{\qquad}r}
topleft&   top& topright\\
left&  center&  right \\
bottomleft& bottom & bottomright\\
\end{tabular}
\end{center}
\end{quote}

\dockey{steps}{n}
\begin{quote}
Make the transition in \docid{n} steps. 
\end{quote}
The optional parameter \doctt{global} indicates that
the definition is global and will affect the following pages, 
as well as the current page. 

By default, the transition affects only the current page. 
\end{quote}

\medskip\noindent
\docdef \advitransbox \docarg {key=val list}\docarg{hbox material}
\begin{quote}
where \docid {key=val list} is as above and \docarg{hbox material} is
whatever can follow an \docdef \hbox{} command. In particular, the material
may contain verbatim commands, since as for the \docdef\hbox{} it is 
parsed incrementally.
\end{quote}

\subsection {Hyperrefs}

Active dvi supports hyperrefs. 

Explain \verb"<a name>aaa</a>"

\subsection {Embed applications anchors}

The commands are 

\medskip\noindent
\docdef \adviembed\docopt{key=value 
list}\docarg{width}\docarg{height}\docarg{command} 

\begin{quote}
where \docid{key=value list} is a list of bindings of the following kind:

\dockey{name}{name}
\begin{quote}
Allows to refer to the embedded application as \docid{name}. 
Annonymous applications have actually the default name \doctt{anonymous}.
\end{quote}

\dockey{ephemeral}{name}

\begin{quote}
This is the default case: the embedded application is killed when the page
is turned.
\end{quote}

\dockey{persistent}{name}

\begin{quote}
The application keeps running in the background, but will only be visible in
the current page.
\end{quote}

\dockey{sticky}{name}
\begin{quote}
The application keep running and remains visible when turning pages.
\end{quote}

\end{quote}

\medskip\noindent
\docdef \advikillembed \docarg{name}
\begin{quote}
Kill all embedded applications named \docid{name}
\end{quote}


\subsection {Active anchors}

The command is 

\medskip\noindent
\docdef \advianchor \docarg{tag}\docarg {text}
\begin{quote}
The text is first display as usual and mark as active. 
Mouving the mouse above the text will display the record \docid{tag}.
\end{quote}

\noindent
The command is also available as an environment 

\medskip\noindent
\docdef \begin\doctt{\{Advianchor\}}\docarg{tag}%
\docid {text}\docdef \end\doctt{\{Advianchor\}}
\begin{quote}
Here, \docid{text} can have fragile commands. 
\end{quote}

\newpage

\subsection {Postscript specials}

\subsubsection {PStricks}

For instance, some features of {\tt
PStricks} such as annotations above arrows do not work, yet.

\subsubsection {PStricks that are know to work}
\begin{itemize}
\item[-]
Labels over arrows \verb"\Aput", \verb"Bput", etc.
(they change the Postscript coordinates...)

\item[-]
{\tt pspicture}
(idem, but drawings that are not embedded in pspictures works)

\item[-]

Advi only recognizes binary overlays (display if 0, hide otherwise). 
This is sufficient for the proper documentclass
(see \hyperlink {overlays}{below}).
\end {itemize}
PS-tricks that are known to work:
\begin {itemize}

\item[+]
Simple drawings work

\item[+] 
{\tt\string\SpecialCoor} works, {\em i.e.} commands of the form
{\tt \string \rput \{A\}\{bla bla\}} works where {\tt A} is a node

\item[+]
Connections between nodes {\tt \string \ncarc}, {\tt \string \ncarc},
also works.
\end {itemize}


\section {Auxiliary packages}

\subsection {Annotations}
\subsection*{Active Annotations (manual)}

To use annotations, you must load the additional package
\begin{verbatim}
\usepackage{advi-annot}
\end{verbatim}
This defines the following command:

\medskip\noindent
\docdef \adviannot \docopt{key=value list}\docarg{anchor}%
    \docopt{ps options}\docpar {pos}\docarg{text}
\begin{quote}
The \docid{key=value list} is a list of bindings of the following kind:

\dockey[yellow]{bg}{color}
\begin{quote}
The background color for annotations. 
\end{quote}

\dockey[c]{col}{colspec}
\begin{quote}
Where \docid{colspec} is a column specification for the tabular
environment. Moreover, the following abbreviations are recognized:

\smallskip
\tt
\begin{tabular}{|l|l|}
\hline
\rm\em key & \rm expands to \\
\hline
c &  col=c\\
l &  col=l\\
r &  col=r\\
p=\docid{w} & col=p\docarg{w}\\
\cline{1-2}
\end{tabular}
\begin{tabular}{|l|l|}
\hline
\rm\em key & \rm expands to \\
\hline
C &  col=\{>\{\$\}c<\{\$\}\}\\
L &  col=\{>\{\$\}l<\{\$\}\}\\
R &  col=\{>\{\$\}r<\{\$\}\}\\
P\docid{w} &  \doctt{col=\{>\{\$\}p\docarg{w}<\{\$\}\}}\\
\cline{1-2}
\end{tabular}
\end{quote}

\docid{pos} is the optional relative position of the annotation 
defaults to $1,1$ (it is passed to the command \docdef\rput).

\docid{ps options} are pased to  the command \docdef\psset). 
\end{quote}
Parameters (color and tabular column specification) can also 
be set globally using the command:

\medskip \noindent
\docdef \setkeys\doctt{\{adviannot\}}\docarg{key=value list}


%Annotation location can also be invisible \advi{C}{\tt{ }}instead of visible.


\appendix


\section {Known bugs}

\subsection*{Inlined postscript}

PS relies on {\tt ghoscript} to display Postscript inlined specials.
However, some earlier releases of {\tt ghostscript} 
implements Postscript {\tt flushpage} command as a {\tt XFlush} 
which does not force evaluations of commands, and make the 
synchronization between {\tt ghostscrip} and {\tt advi} drawings 
uncontrolable. In this case, the inerleaving of inlined postscript 
and other material may be inconsistent. 

Fortunately, it seems that recent version of ghostscript $(> 6.5)$ have
fixed this problem by using {\tt XSync(false)} instead.
There, inline specials should be correctly rendered. 

However, version some releases of 6.5x also carry a small fatal bug for
advi, that will be fixed in future releases. This can easily be fixed, thoug 
(see \url{http://cristal.pauillac.inria.fr/~remy/ghostscript/}).

\subsection*{Inlined postscript}

So far, the implementation of inline Postscript does not correctly handle
complex change of coordinates.  (See PStricks section). 


\section {Future}

\section {Key bindings}

\section {Index}

\printindex

\end{document}
